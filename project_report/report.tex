\documentclass[a4paper]{article}

\title{Project report: Formal Methods for Information Security}
\author{Sabina Fischlin \& Wanja Chresta}

\begin{document}
\maketitle

\newcommand{\Fr}{\texttt{\textasciitilde{}Fr}}
\newcommand{\LTK}{\texttt{!LTK}}

\section{PACE Protocol}
\subsection{Assignment 1.1, Theory P1}
For this theory, we use the signing builtin to be able to create a MAC. We setup symmetric keys using a \Fr{} value that goes into \LTK .
\subsubsection{Alice and Bob}
\begin{eqnarray*}
A \to B        & : & x\\
A \leftarrow B & : & [x]_{k(A,B)}
\end{eqnarray*}
\subsubsection{Results}
We were able to prove {\em noninjective} and {\em injective agreement} for the initiator. We have been also able to disproove them for the responder (obviously).

\subsection{Assignment 1.2, Theory P2}
We interleave two P1 protocols with eachother, as described in the assignment. The actors have seperate symmetric shared keys $k(A,B)$ and $k(B,A)$ in their initial state.
Have defined the 
\subsubsection{Alice and Bob}
\begin{eqnarray*}
A \to B        & : & x\\
A \leftarrow B & : & y\\
A \to B        & : & [y]_{k(A,B)}\\
A \leftarrow B & : & [x]_{k(A,B)} 
\end{eqnarray*}
\subsubsection{Results}
...

\subsection{Assignment 1.3, Theory P3}
...
\subsubsection{Alice and Bob}
\begin{eqnarray*}
...
\end{eqnarray*}
\subsubsection{Results}

\subsection{Assignment 1.4, Theory P4}
...
\subsubsection{Alice and Bob}
\begin{eqnarray*}
...
\end{eqnarray*}
\subsubsection{Results}
...

\subsection{Assignment 1.5, Theory P5}
c) relies on the secrecy for a guard against MitM attack. Same as could be done on DH. Need the secrecy since g^y and g^x are transmitted as plaintext.

\end{document}